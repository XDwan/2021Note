\documentclass{article}

\usepackage{lmodern}
\usepackage[UTF8]{ctex}% 中文语言包
\usepackage{amsmath}
\usepackage{booktabs}% 引入三线表宏包
\usepackage{indentfirst}% 使用indentfirst宏包
\setlength{\parindent}{2em}% 设置首行缩进距离
\usepackage{graphicx}% 插入图片宏包
\usepackage{float}% 图片错位解决

\title{组合数学}

\author{XDwan}

\begin{document}
    
\renewcommand{\contentsname}{目录}%将content转为目录
\tableofcontents
\newpage

\section{绪论}

\subsection{三个问题}

\begin{enumerate}
    \item \textbf{存在性问题} 解是否存在
    \item \textbf{计数问题} 有多少种解
    \item \textbf{构造问题} 解应该怎么构造
\end{enumerate}

\subsection{求解方法}

\begin{enumerate}
    \item \textbf{数学归纳法}
    \item \textbf{迭代法}
    \item \textbf{一一对应技术} 
    
    利用两个事物间一一对应的关系,将较复杂的组合计数问题A转化成容易计数的B

    \item \textbf{殊途同归方法} 
    
    从不同角度讨论计数问题,建立组合等式

    \item \textbf{数论方法} 
    
    利用整数的奇偶性、整除性等数论性质进行分析推理


\end{enumerate}

\subsection{两个法则}

\begin{enumerate}
    \item \textbf{加法法则} 
    
    若完成一件事有两个方案,第一个方案有m种方法,第二种有n种,只要选择任何方案中某一种方法就能完成,并且方法两两互不相同,则完成事情共有m+n个方法
    
    \item \textbf{乘法法则} 
    
    若完成一件事有两个步骤,第一步有m个方法,第二步有n种方法,则完成事情共有 m $\cdot$ n 种方法

\end{enumerate}

\subsection{排列与组合}

\subsubsection{相异元素不允许重复的排列数和组合数}

从n个相异元素种不重复的取r个元素的排列数和组合数分布为:

\begin{equation}
    P^r_n=P(n,r) = n(n-1)\dots(n-r+1) = \frac{n!}{(n-r)!}
\end{equation}

\begin{equation}
    C^r_n = C(n,r) = \begin{pmatrix}
        n \\ r
    \end{pmatrix} = \frac{P^r_n}{r!} = \frac{n!}{(n-r)!r!}
\end{equation}

相异元素不允许重复的排列问题也可以描述为:将r个有区别的球放入n个不同的盒子,每个盒子不超过一个,则总的放法数为$P(n,r)$。同样,若球不加区别,则有$C(n,r)$种放法。这是排列与组合的数学模型——\textbf{分配问题},也称\textbf{分配模型}


\subsubsection{相异元素允许重复的排列}

从n个不同元素种允许重复的选r个元素的排列,简称\textbf{r元重复排列}。其排列个数记为$RP(\infty,r)$。其对应的分配模型为将r个有区别的球放入n个不同的盒子,每个盒子的球数不加限制而且同盒的球不分次序。则排列数为:
\begin{equation}
    RP(\infty,r) = n^r
\end{equation}

从集合的角度理解,问题也可以描述为:

设集合$S= \{\infty \cdot e_1,\infty \cdot e_2,\dots,\infty \cdot e_n \}$,即$S$种共含$n$类元素,每类元素有无穷多个,从$S$中取$r$个元素的排列数即为$RP(\infty,r)$

\subsubsection{不尽相异元素排列}

设$S=\{n_1\cdot e_1,n_2\cdot e_2,\dots,n_t\cdot e_t\}$,即元素$e_i$有$n_i$个$(i=1,2,\dots,t)$,且$n_1+n_2+\dots+n_t = n$,从$S$中任取$r$个元素,求其排列数$RP(n,r)$

本问题的数学模型事将$r$个有区别的球放入$t$个不同的盒子,而每个盒子的容量事有限的,其中第$i$个盒子最多只能放入$n_i$个球,求分配方案数。

相对于前两种情况,这种事有限重复的排列问题,即相异元素不重复的排列强调事不重复,即盒子的容量为1;允许重复的排列实际上是盒子容量无限。而有限重复是盒子容量有限。

如下是几个特例:

\begin{enumerate}
    \item r=1 时,$RP(n,1)=P^1_t = t$
    \item r = n 时,n个元素的全排列数为

    \begin{equation}
        RP(n,n) = \frac{n!}{n_1!n-2!\dots n_3!}
    \end{equation}

    \item t = 2 时, 2个元素的全排列数为
    
    \begin{equation}
        RP(n,n) = \frac{n!}{n_1!n_2!} = \begin{pmatrix}
            n \\ n_1
        \end{pmatrix}
    \end{equation}
\end{enumerate}

\end{document}